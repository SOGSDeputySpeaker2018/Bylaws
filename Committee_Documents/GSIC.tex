\section{Graduate Student Issues Committee}

\subsection{Ratified}
August 22, 2019

\subsection{Mandate}
As per Bylaw 2.9.4.1, the Graduate Student Issues Committee (GSIC) is a standing committee (Bylaw 2.9.5.10) of the Society, the mandate of GSIC is to monitor, assess, and respond to issues pertaining to the quality and accessibility of graduate education.

\subsection{Article I: Membership}
\begin{longenum}[ label*=\thesubsection.\arabic*., align=left] 
\item The Graduate Student Issues Committee (GSIC) shall have:
	\begin{longenum}[ label*=\arabic*., align=left]
	\item An Official Liaison, who shall be the Vice-President Advocacy, and who shall be a non-voting member of the committee in compliance with Bylaw 2.9.3.2 and 2.9.3.3.
	\end{longenum}
\item The GSIC will elect:
	\begin{longenum}[ label*=\arabic*., align=left]
	\item A Chairperson, who shall:
		\begin{longenum}[ label*=\arabic*., align=left]
		\item be elected from amongst the committee members;
		\item act in compliance with the Duties of the Chairperson (Bylaw 2.9.2);
		\item and fulfill the duties in Bylaw 2.9.5.11.1;
		\end{longenum}
	\item A Deputy Chairperson (if applicable), who shall assume the duties of the Chairperson at the discretion of the Chairperson or the committee or in the absence of the Chairperson.
	\end{longenum}
\item A membership limit, as per Bylaw 5.5.3.1.3., set at 13 voting members, including the Chairperson.
\item All voting members of the GSIC shall be full and associate members of the Society nominated to the committee by the Society’s Council.
\end{longenum}

\subsection{Article II: Committee Management}
\begin{longenum}[ label*=\thesubsection.\arabic*., align=left] 
\item The GSIC will act in accordance with Bylaw 2.9.3.5 and 2.9.3.7 to determine quorum, Bylaw 2.9.3.9 to manage and maintain attendance, and committee membership, and Bylaw 2.9.4.2.1 to count votes.
\item Voting Policies
	\begin{longenum}[ label*=\arabic*., align=left]
	\item Aside from the scheduled meetings, online voting is also permitted on the condition that it is for correcting spelling, grammar, or wording mistakes on motions. Voting members shall be given one (1) business day to respond to an online vote, at which point the vote shall be considered concluded. If any voting member objects to business being voted on online, the voting shall be prohibited for that particular piece of business.
	\item Committee members (including the Official Liaison, Chairperson, Deputy Chairperson, and ex-officio members) can hold up to three (3) proxies.
	\end{longenum}
\item Non-committee members may attend as non-voting guests if invited by the chair. Committee members may also request that non-committee members attend as non-voting guests.
\item As per Bylaw 1.6.5. the Chairperson and voting members may serve no more than 12 consecutive months without being ratified for a new council year.
\item Respectful Abandonment: In the event of the failure of a voting member of the committee to attend two meetings during their term either in person or by proxy, the Chairperson may remove the member from the committee in accordance with the protocols listed in Bylaw 2.9.3.9.
\end{longenum}

\subsection{Article III: Operations}
\begin{longenum}[ label*=\thesubsection.\arabic*., align=left] 
\item The GSIC shall maintain a record of issues pertaining to graduate students through an online survey.
\item The GSIC shall research the causes, possible solutions and recommendations for the issues found from the survey.
	\begin{longenum}[ label*=\arabic*., align=left]
	\item The committee will share findings with the student body (SOGS members) at events and on the SOGS website throughout the academic year. 
	\item The committee would survey the student body if all the issues collected from the previous survey have been looked into. Else, the committee can only survey students at maximum of once every three (3) years. 
	\end{longenum}
\item The committee shall follow the term cycle of Vice-President Advocacy as per Bylaw 2.8. At the end of each cycle, the Chairperson is responsible, with the aid of committee members, for drafting a document outlining: 
	\begin{longenum}[ label*=\arabic*., align=left]
	\item the issues tackled during the committee’s research cycle;
	\item the efforts done towards resolving the issues or stating alternatives to accommodate graduate students;
	\item any unresolved issues;
	\item recommendations to the Vice-President Advocacy for continual work on unresolved issues or issues unmet during the committee’s cycle.
	\end{longenum}
\item In accordance with Bylaw 2.9.5.10.4, the GSIC shall select non-executive members of the Society to be dispatched as delegates to general meetings of the Canadian Federation of Students.
\end{longenum}