\section{Discipline Code}

\subsection{Introduction and Purpose}
\begin{enumerate} [align=left]
\item  The purpose of the Society's Discipline Code is to define the general standard of conduct expected of members of the Society of Graduate Students (henceforth known as the Society), provide examples of behaviour that constitutes a breach of this standard of conduct and set out the disciplinary procedures that the Society will follow.
\item The Society is a community of graduate student members involved in learning, teaching, research, and other activities. The Society provides an environment of free and creative inquiry within which critical thinking, mutual respect, and practical skills are cultivated and sustained. It is committed to a mission and to principles that will foster excellence and create an environment where its students and staff can grow and flourish.
\item Members assume the rights and responsibilities associated with membership in the Society's academic and social community. The privileges granted to each member are conditional upon the fulfilment of this responsibility and members must familiarize themselves with Society regulations and the conduct expected of them while studying at Western University.
\item Members are reminded that they are equally responsible for observing the standard of conduct set out in this Code when using any electronic devices to send or post messages or material.
\item The Society encourages informal resolution of minor incidents.
\item Nothing in this Discipline Code shall be construed to prohibit peaceful assemblies and demonstrations, lawful picketing, or to inhibit free speech as guaranteed by law.
\item Any member found responsible for misconduct is subject to the Society's disciplinary sanctions, regardless of the action or inaction of civil authorities. Nothing in this Code precludes the Society from referring an individual matter to Western University or an appropriate law enforcement agency either before, during, or after disciplinary action is taken by the Society under this Code. A member may be subject to criminal prosecution and/or civil proceedings notwithstanding, and in addition to, disciplinary action taken by the Society against the member under this Code.
\end{enumerate}

\subsection{Definitions}
In this code
\begin{enumerate} [align=left]
\item "Member" is an individual that fulfils any one of the requirements of Section 1.5 of the Society's Constitution.
\item "Ombudsperson" is the Society's Ombudsperson (Bylaw 2.2.7).
\item "Appeals Review Commission (ARC)" is the Commission that is defined under the Society's Bylaw 2.4.2.
\item "Premises of the University or its Affiliated University Colleges" shall include lands, buildings, and grounds of the University and its Affiliated University Colleges and other places or facilities used for the provision of the University's courses, programs, or services.
\item "A Society sponsored program, event, or activity" shall be interpreted as a program, event, or activity that is hosted, sponsored, or organized by the Society and includes, but is not limited to organized trips, the Grad-Club, Council and Committee Meetings, and the Western Research Forum.
\end{enumerate}

\subsection{Relationship to Other University Policies and Codes}
\begin{enumerate} [align=left]
\item If a member's conduct could be considered a breach of this Code and also a breach of its Conflict of Interest Bylaw and Policy, the Society may proceed under the Code or under the Conflict of Interest Bylaws and Policies. A member may not be penalized under both the Code and these Policies for the same conduct.
\end{enumerate}

\subsection{Jurisdiction}
\begin{enumerate} [align=left]
\item This Code applies to
\begin{enumerate} [label*=\arabic*., align=left]
\item Conduct that occurs on the premises of the University or its Affiliated University Colleges;
\item Conduct that occurs at a Society sponsored or sanctioned program, event, or activity, whether the program, event, or activity is on campus or off-campus; and
\item Other off-campus conduct, for example,
\begin{enumerate} [label*=\arabic*., align=left]
\item When the individual is acting as a representative of the Society. \item That has, or might reasonably be seen to have an adverse effect on, interfere with, or threaten the proper functioning of the Society, its mission, the rights of a member of the Society community to use and enjoy the University's learning and working environments, or that raises concerns for the safety or security of an individual or individuals while on campus or while participating in Society programs, events or activities.
\end{enumerate}
\end{enumerate}
\item Members are subject to the provisions of this Code except when acting in their capacity as Graduate Teaching Assistants or Research Assistants.
\end{enumerate}

\subsection{Misconduct and Prohibited Conduct}
The following list sets out specific examples of prohibited conduct. This list is illustrative only and is not intended to define misconduct in exhaustive or exclusive terms.
\begin{enumerate} [align=left]
\item Disruption: By action, threat, written material, or by any means whatsoever, disrupting or obstructing any Society activities, including a Society sponsored or sanctioned program, event, or activity, or other authorized activities on premises of the University or its Affiliated University Colleges, or the right of another person to carry on his/her legitimate activities, or to speak or to associate with others. The Society activities include, but are not limited to, research, studying, sports and recreation, administration, and meetings.
\begin{enumerate} [label*=\arabic*., align=left]
\item Misconduct against persons and dangerous activity, such as:
\item Any assault, harassment, intimidation, threats or coercion.
\item Conduct that threatens or endangers the health or safety of any person.
\item Contravention of the University's Non-Discrimination/Harassment Policy. 
\item Knowingly (which includes when one should reasonably have known) creating a condition that endangers the health, safety, or well-being of any person.
\item Engaging in conduct that is, or is reasonably seen to be, humiliating or demeaning to another person or coercing, enticing or inciting a person to commit an act that is, or is reasonably seen to be, humiliating or demeaning to that person or to others.
\end{enumerate}
\item Misconduct involving property, such as:
\begin{enumerate} [label*=\arabic*., align=left]
\item Unauthorized entry and/or presence on any premises of the Society or any premises used for Society sponsored or sanctioned programs, events or activities.
\item Misappropriation, damage, unauthorized possession, defacement and/or destruction of premises or property of the Society, or the property of others.
\item Use of the Society facilities, equipment or services contrary to express instruction or without proper authority.
\end{enumerate}
\item False Information, such as:
\begin{enumerate} [label*=\arabic*., align=left]
\item Furnishing false information.
\item Forging, altering or misusing any document, record, card or instrument of 
identification.
\end{enumerate}
\item Violation or contravention of the Society's written Policies, Regulations, or Rules
\item Contravention of other laws, such as:
\item Contravention of any provision of the Criminal Code or any other Federal or Provincial statute or Municipal by-law.
\item Aiding or encouraging others in the commission of an act prohibited under this Code or attempting to commit an act prohibited under this Code.
\item Failure to comply with any sanction imposed by the Society for misconduct under this Code.
\end{enumerate}

\subsection{Interim Measures}
\begin{enumerate} [align=left]
\item Temporary Exclusions. The Official Liaison to a committee may exclude a member from said committee if the Official Liaison believes on reasonable grounds that the member's continued presence is detrimental to good order or constitutes a threat to the safety of others. Such initial exclusion shall last for the duration of the meeting. The Official Liaison shall report the exclusion immediately to the Ombudsperson.
\item Interim Prohibition. The Ombudsperson may impose an interim prohibition pending an investigation and disposition of a complaint of misconduct. Interim prohibition may be imposed only:
\begin{enumerate} [label*=\arabic*., align=left]
\item If needed to ensure the safety and well-being of members of the Society's community or preservation of the Society's property;
\item If needed to ensure the Member's own physical or emotional safety and well-being; or
\item If there is a reasonable apprehension that the member poses a threat of disruption or of interference with the normal operations of the Society.
\end{enumerate}
\item During a period of interim prohibition, a member may be denied access to specified Society facilities (including the Grad Club) and/or any other Society sponsored and/or sanctioned activities or privileges for which the member might otherwise be eligible, as the Ombudsperson may determine to be appropriate. Within two (2) working days following the imposition of interim prohibition, the member shall be informed in writing of the reasons for the prohibition. The member shall be afforded the opportunity to respond to the allegations being made against them. If the member responds, the Ombudsperson will reassess the prohibition and either revoke or continue the prohibition pending formal disposition of the matter.
\end{enumerate}

\subsection{Complaint Procedures}
\begin{enumerate} [align=left]
\item Any member(s) or employee(s) may submit a complaint of misconduct against any member(s). A complaint should be submitted to the Society's Ombudsperson.
\item The Ombudsperson shall not make a finding of misconduct nor impose a sanction or sanctions against a member unless the member has been informed, in writing, of the nature of the complaint, the facts alleged against them, and has been given a reasonable opportunity to respond and to submit relevant information. The member shall also be given a reasonable opportunity to meet personally with the Ombudsperson to discuss the matter. It is the responsibility of both parties to provide all materials and information that will support their positions. Furthermore, the Ombudsperson will make reasonable attempts to ascertain the truth to the best of their ability.
\begin{enumerate} [label*=\arabic*., align=left]
\item If the Ombudsperson feels that they are not in a position to fairly rule on a complaint due to conflict of interest or must recuse themselves for any reason, they can forward the complaint to a member of the Appeals Review Commission (ARC), chosen at random, by the Speaker, to act as the Ombudsperson for this matter only.
\end{enumerate}
\item If the Ombudsperson concludes that there has been misconduct, the Ombudsperson may impose an appropriate sanction or sanctions.
\item If the member does not respond to the allegation or does not meet with the Ombudsperson after having been given a reasonable opportunity to do so, the Ombudsperson may proceed to dispose of the complaint without such a response or meeting.
\item At all meetings with the Ombudsperson, both parties may be accompanied by a colleague of their choosing. Legal representation is not permitted at this stage; it is only permitted at the appeal stage. \item In determining an appropriate sanction or sanctions, the Ombudsperson may take into account any previous findings of misconduct. The Ombudsperson may direct that a sanction be held in abeyance if a member's registration at the University is interrupted for any reason.
\item The decision of the Ombudsperson, with reasons, shall be communicated in writing (electronic media are acceptable) to the member. If there is a finding of misconduct, a copy of the decision will be retained in the Society's Administrative Office. A copy of the decision shall be provided on a need-to- know basis to administrative units (e.g. Executive Officers and Non-Executive Officers). The Speaker and the relevant Executive is responsible for the implementation of any decision made under the Code.
\item All notices and other communications from the Ombudsperson to the member in question or any other member of the University community shall be by personal delivery, campus mail, e-mail, priority post, courier, or registered mail.
\item Complaints of misconduct shall be reported, investigated, and decided in a timely manner. 
\item The Ombudsperson shall report annually at the Annual General Meeting (AGM), summarizing the number of complaints received, number of complaints investigated, and the general nature of the matters investigated.
\item After five (5) year the files will be expunged.
\end{enumerate}

\subsection{Appeals}
\begin{enumerate} [align=left]
\item A member may appeal an Ombudsperson's finding of misconduct to the Appeals Review Commission (ARC) on one or more of the following grounds:
\begin{enumerate} [label*=\arabic*., align=left]
\item There was a serious procedural error in the hearing of the complaint by the Ombudsperson which was prejudicial to the appellant;
\item New evidence, not available at the time of the earlier decision, has been discovered, which casts doubt on the decision;
\item The Ombudsperson did not have the authority under this Code to reach the decision
or impose the sanctions.
\end{enumerate}
\item Filing an Appeal Application will not stay the implementation of any sanctions imposed. 
\item The Appeals Review Commission (ARC) may:
\begin{enumerate} [label*=\arabic*., align=left]
\item Deny the appeal;
\item In the case of an appeal under 3.2.8.9.1 or 3.2.8.1.2, grant the appeal and direct the Ombudsperson to re-hear the matter or reconsider some pertinent aspect of their decision, and may include recommendations relating to the conduct of any rehearing, or repeal the original decision;
\item In the case of an appeal under 3.2.8.1.3, grant the appeal and repeal the original decision.
\end{enumerate}
\item The right to be represented by legal counsel will be accorded to the principal parties to the appeal at this level.
\item The parties must bear all their own legal expenses, if any.
\item Appeal Procedures
\begin{enumerate} [label*=\arabic*., align=left]
\item An Appeal Application must be filed with the Society's office staff in a sealed envelope addressed to the ARC Chair within two weeks after a decision has been issued by the Ombudsperson.
\item An Appeal Application will include the following items:
\begin{enumerate} [label*=\arabic*., align=left]
\item Name, student number, email address, current phone number, and current mailing address of the Appellant.
\item Date of the Original Notice of Decision.
\item Reason/Grounds for Appeal (new evidence, for example).
\item Explanation of Disagreement with the original decision with reasoning, including any relevant documentation and outcome(s) sought.
\begin{enumerate} [label*=\arabic*., align=left]
\item Where the basis of the appeal is new evidence, such new evidence shall be described comprehensively and the names of any witnesses shall be provided.
\end{enumerate}
\item Name of legal counsel or agent (if any).
\item Any accessibility issues impacting the Appellant’s participation in the Appeals Review Panel Hearing.
\item In signing the Appeal Application, the Appellant acknowledges that they must appear before an Appeals Review Panel, understands the policies of the Appeal process (as described below), and is beholden to the verdict of the Appeals Review Panel.
\end{enumerate}
\item The ARC Chair must inform the other members of the Commission and the Ombudsperson that an Appeal Application has been received.
\item An Appeal Application will not be accepted by the ARC Chair if incomplete or not filed within the time period specified in 3.2.8.7.1. Exceptions to the time limit for filing an Appeal Application are at the discretion of the ARC upon written application of the member.
\item Parties in attendance at to an appeal are the ARC Panel and Chair, the member against whom the decision has been made (Appellant) and the Ombudsperson (Respondent). 
\item The Respondent (Ombudsperson) shall file a concise written reply to the appeal with the Speaker within five (5) business days of receiving the documents. A copy of the reply shall be provided to the Appellant.
\item Upon receipt of an Appeal Application, the ARC Chair shall:
\begin{enumerate} [label*=\arabic*., align=left]
\item Constitute a Panel of at least three (3) members including the chairperson. If the chairperson recuses themself from the proceedings, the panel must elect an interim chairperson from within their ranks.
\item Member(s) from the same department as either of the parties shall recuse themselves from the panel.
\item Facilitate the scheduling of the initial meeting of the Panel.
\end{enumerate}
\item The Panel is bound by neither strict legal procedures nor strict rules of evidence. It shall proceed fairly in its disposition of the appeal, ensuring that both parties are aware of the evidence to be considered, are given copies of all documents considered by the Panel, and are given an opportunity to be heard during the process.
\item The Panel may summarily dismiss an Appeal Application if it does not, in the judgement of the Panel, raise a valid ground of appeal or does not assert evidence capable of supporting a valid ground.
\item The Panel shall hold an oral hearing if any party involved and/or member(s) of the ARC Panel requests one.
\begin{enumerate} [label*=\arabic*., align=left]
\item. Both parties may petition the ARC Panel to make the oral hearing open to the Society or in-camera.
\item The decision to make an oral hearing open to the Society rests with the ARC Panel.
\item While an attempt shall be made to schedule an oral hearing at a time convenient to the Panel and the parties, a request by a party for a lengthy delay in the scheduling of the hearing, or a postponement of a scheduled hearing, will be granted by the Chair only in exceptional circumstances. Oral hearings will ordinarily be held within six (6) weeks of filing of the appeal. In the case of an oral hearing, if the ARC Chair is unable to contact the Appellant within a reasonable time (i.e., ten [10] business days) to schedule a hearing, the Appellant will be notified by registered mail at the address on the Appeal Application of the deadline by which they must contact the ARC Chair to arrange a hearing. If the Appellant has not contacted the ARC Chair by the specified deadline, the appeal will be considered to be abandoned. 
\item Each party to an oral hearing shall be sent a Notice of Hearing, setting out the time, place, and purpose of the hearing. If a party does not attend, the Panel may proceed in the party's absence.
\end{enumerate}
\item Each member of a Panel, including the Chair, shall vote. There shall be no abstentions. A majority of positive votes is required to grant an appeal.
\item The decision, with reasons, shall be filed with the Speaker and copies shall be sent to the parties present at the proceedings as well as to others with a legitimate need to know (e.g. Relevant Executive)
\end{enumerate}
\end{enumerate}

\subsection{Review of Code}
The Policy Committee shall complete a review of the Code at the start of the Ombudsperson's term.
