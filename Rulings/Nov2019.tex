\begin{center}
\rule{\textwidth}{3.6pt}\\[\baselineskip] % Thick horizontal linee
\begin{Huge}
\textbf{Speaker's Ruling}
\end{Huge}

\rule{\textwidth}{3.6pt}\\[\baselineskip] % Thick horizontal line



\vspace*{2\baselineskip} % Whitespace Between Title and Discriptive Title
\end{center}
\section*{Prorogation of the 2018-2019 Council Year}

\index{Speaker!Rulings!Absenteeism on Committees}
Dear Councillors,

A non-zero number of Councillors asked about the proceedings of the September Council meeting, expressing confusion and a request for clarification on business that was conducted and unfinished business due to a loss of quorum. My ruling follow. 

	\begin{multicols}{2}
Given that October marks the start of the SOGS council year, I am compelled to prorogue the previous session and proclaim a new council session for the October 2019 meeting.
My proposal for prorogation is for two reasons:  

i) to prevent confusion and to help new (and returning) Councillors at the start of a new council year by presenting a clean/clear agenda for the meeting.

ii) to allow for all presented motions from September Council (included in the package and from the floor) to be resubmitted under Orders of the Day such that motions could be updated, revised, withdrawn, or discussed in consultation with parties affected for language or anticipated execution.

Starting the new council year in a state of tabula rasa provides an opportunity for further review of the motions on the previous agenda before resubmitting. Ideally, this will help clarify proposed motions and streamline debate to the actual change proposed in the motion as opposed to, say, adjusting the language and punctuation from a grammatical perspective.

The conclusion of the previous council session is not an unfamiliar practice. In this context, it also does not prevent previous business from being discussed in the new session. My objective in this ruling is not to erase the motions put forth, but to allow motions to be submitted to the package as a fresh session. Doing so would allow Councillors the opportunity to review motions and associated contents. Ultimately, this decision will hopefully quell any confusion for all council members.

\end{multicols}


\vskip 2cm
\noindent
Respectfully yours, \newline
\noindent
Lola Wong \newline
\indent
Speaker, \newline 
\indent
Society of Graduate Students \newline
\indent
speaker@sogs.ca \newline
\indent 
November 2019