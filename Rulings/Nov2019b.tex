\begin{center}
\rule{\textwidth}{3.6pt}\\[\baselineskip] % Thick horizontal linee
\begin{Huge}
\textbf{Speaker's Ruling}
\end{Huge}

\rule{\textwidth}{3.6pt}\\[\baselineskip] % Thick horizontal line

\vspace*{1\baselineskip} % Whitespace Between Title and Discriptive Title
\end{center}
\section*{Motion ruled out of order at October Council, 2019}

\index{Speaker!Rulings}
Honourable Councillors,

In accordance with bylaw 2.2.2.3., I present this as documentation of my ruling made during Council on October 24, 2019. Rather than regurgitate the event, I refer you to the October 2019 Minutes of the November 2019 Council package for details on the proceedings.
This is simply a written summary and reflection on the ruling. 

	\begin{multicols}{2}
\subsection*{Background:}
A point of order was raised at the October 2019 Council meeting concerning a motion submitted by the SOGS Executive requesting Council’s approval for funds to be drawn from the SOGS Reserve as additional compensation for their increased workload related to the Student Choice Initiative.
The point of order was made by the SOGS Ombudsperson, who was also acting as an alternate for a Councillor from his department. The point raised was that the motion was in conflict with bylaw 2.12.6 and was out of order. A sidebar was called, and the information was presented and discussed in a small group with the SOGS Ombudsperson, acting Deputy Speaker, SOGS President, SOGS VP Finance, and myself. The SOGS Ombudsperson indicated that Bylaw 2.12.6 states that all new spending must be approved/recommended by the Finance Committee. The motion from the Executive specified that the recommendations were from the Finance Co-Chairs. Based on the information presented to me at that time, I ruled the motion out of order.
Council proceedings resumed, I announced my ruling and the VP Advocacy challenged the ruling. Numerous questions and points and were raised by the assembly during the debate on the challenge of the ruling, and after deliberation, a motion to call to question was made and passed. The assembly voted in favour of the ruling and the challenge was defeated.

\subsection*{The Motion submitted by the SOGS Executive}
Whereas the executives have been working unrealistic hours in light of the Student Choice Initiative (SCI);\newline
Whereas the Executive provided a document outlining the rationale and justification of this request to the Finance Committee Co-chairs and is available in the October Council Package Appendix;\newline
Whereas the Finance Committee Co-chairs have been consulted;\newline
Whereas the Finance Committee Co-chairs support this request;\newline
BIRT Council approves the executive compensation for the increased workload related to the SCI of \$3,560 per executive;\newline
BIFRT the compensation be drawn from SOGS Reserve. 

\subsection*{The Bylaw cited by the SOGS Ombudsperson}
2.12.6. Proposals for amendments to the budget shall be received by the Vice-President Finance and shall be referred to the Finance Committee. The Finance Committee shall present the proposals with the Finance Committee’s recommendations to Council within eight weeks of the Vice-President Finance’s receipt of the proposals. 

\subsection*{The Ruling}
The point of order brought attention to the bylaw 2.12.6. states that amendments to the budget shall be referred to the Finance Committee. Subsequently, the Finance Committee is to present the proposals with recommendations to Council. The argument presented was that the bylaw implies budgetary changes —or changes in the budgetary spending— should be recommended to Council by the Finance Committee. The Executive specify consultation with the Co-Chairs in their motion but not with the Finance Committee.

When the ruling was challenged, the question of past precedent was raised. While previous motions or rulings serve as important guides, every situation and context is different and must be examined on its own merits. I made a ruling based on the information presented at that moment and chose what I believed was the right decision. I accept that others will disagree with that and many other of my decisions, but without debate and discourse, collaboration and communication, our Society will not evolve.

The Challenge of the Speaker’s Ruling was, in fact, paramount. It was an opportunity for Councillors to choose what they believed was the right decision for that moment. More importantly, it was an excellent demonstration of the power Councillors have as the governing body of SOGS. I will continue to keep things engaging, ensure that everyone is informed, and endeavour to create those opportunities for all of the members of SOGS in future meetings to the best of my ability.


\end{multicols}


\vskip 2cm
\noindent
Respectfully yours, \newline
\noindent
Lola Wong \newline
\indent
SOGS Speaker, \newline 
\indent
speaker@sogs.ca \newline
\indent 
November 2019