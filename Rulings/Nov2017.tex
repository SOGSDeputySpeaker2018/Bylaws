%\setcounter{section}{0}
\index{Absenteeism!Committee!Speaker's Ruling|(}
\begin{center}
\rule{\textwidth}{3.6pt}\\[\baselineskip] % Thick horizontal linee
\begin{Huge}
\textbf{Speaker's Ruling}
\end{Huge}

\rule{\textwidth}{3.6pt}\\[\baselineskip] % Thick horizontal line



\vspace*{2\baselineskip} % Whitespace Between Title and Discriptive Title
\end{center}
\section*{Removing Absentee Committee Members from Committees}

\index{Speaker!Rulings!Absenteeism on Committees}
Dear Councillors,

The President our Society has asked me the following questions and asked for a ruling on them: 
\begin{longenum}[ label*=\arabic*., align=left]

\item A committee chair wanted to know what to do in the situation where the committee can't meet quorum, and therefore can't vote on the removal of members who have missed two meetings without assigning a proxy, received a warning, and missed a third meeting. 

\item A conversation surrounding the issue of committee members who graduate or move away led those present to wonder if the following course of action would be acceptable:
\begin{longenum}[ label*=\arabic*., align=left]
\item An email asking the member if they wish to continue on the committee.
\item If no response is received, a second email asking if the member wishes to continue, with something to the effect of "no response will be understood as your resignation from the committee."
\item Removal of the member from the committee list following their understood resignation.
\end{longenum}



\end{longenum}	

	\begin{multicols}{2}
\subsection*{First Question:}
The first question has a straight-forward answer.  The bylaws as they currently exist do provide 3 non-mutually exclusive remedies for Committee chairs for dealing with absentee committee members in the absence of quorum being established or possible at the committee level: Bylaw 2.9.3.9, Bylaw 2.7 and expiration of 1 year term on the committee.  

An option at the disposal of the committee chair is to recruit new members to the committee to such a number that they can forge a new quorum and remove the absentee committee members under 2.9.3.9.  The speaker interprets the current text of 2.9.3.9 as a simple majority vote once the threshold of absenteeism is achieved.   \index{Committee!Removal!Absenteeism} 

The other option would be to use the recall provisions under bylaw 2.7. For sake of clarity, the process unfolds as follows:  
\begin{longenum}[ label*=\arabic*., align=left]
\index{Committee!Recall!Procedure}

\item A committee member \footnote{most likely the Committee Chair since they are the person most knowledgeable on the attendance record of committee members} writes to the Speaker with a recall petition\footnote{For committee members, the petition has no length requirement, and as such is effectively a petition with one name- that of the complainant} and rational for why a member should be removed from a committee (In this case absenteeism without providing a proxy)\footnote{Due to timeline requirements in Section 2.7.1.3, the deadline for addressing this in the next council meeting is the third to last Tuesday of the month.}
\item Within 2 business days of receiving the petition, the Speaker forwards the rational to the committee member facing recall. The member has 3 business days to respond and have that response included as part of the recall motion in the council mailout\footnote{N.B. The Speaker must under 2.7.1.5. remove any libellous material that goes in the council mail out}.  
\item The motion is addresses as part of regular business, and as such must be seconded. \footnote{While not an explicit requirement, the speaker feels that should the committee member being recalled is not a regular member of council, the Speaker should grant them an invitation to council in order to argue their case.}  
\item Assuming the motion is seconded, the complainant and the seconder motivate the motion, and debate follows.
\item The vote is conducted by roll-call.
\item Provided that the motion takes place before elections, the removed member may re-apply to sit on the committee.  
\end{longenum}  

Lastly, a rational behind the introduction of 1 year mandates for committee members a few years ago was to create a passive mechanism for removing absentee committee members from committees. 


\subsection*{Second Question}

The second question is subdivided into 3 parts and this speaker's ruling will deal with the question as such. 
\subsubsection*{Part 1}
With regards to the first act, emailing the committee member and asking if they wish to stay on the committee, the speaker sees nothing wrong with this course of action. 

\subsubsection*{Part 2}
With regards to the second the Speaker finds this very problematic, since the end result of this course of action actively assumes that a non-response to a question of continued participation in a committee is equivalent to consenting to resign. Furthermore, as detailed in the first question, there exists alternative mechanisms for removing a member of a committee who commits chronic absenteeism, and does so in a way that does not infringe on their right to consent or not with regards to resigning.    

Whilst the Bylaws or Robert's Rules of Order Newly Revised, 11th Edition (RONR) do not clearly either allow or prohibit this practice, I will continue the precedent of my predecessor in their November 2014\footnote{Alternate Councillors for Unassigned Council Seats, November 2014, Speaker Chris Shirreff} ruling of using principles to decide the matter.  

When faced with two courses of action in which one infringes on a member's right to consent or to refuse consent, the Speaker must side on the course of action that protects member's rights to consent to a course of action.  As such, the Speaker rules that the assumption that non-response is equivalent to consenting to resign is not in keeping with SOGS values, and thus should not be allowed to take place in our Society.  
	
\subsubsection*{Part 3}
This sub-question becomes moot in light of the ruling on previous sub-question. 

\end{multicols}


\vskip 2cm
\noindent
Respectfully yours, \newline
\noindent
Martin R. Lefebvre \newline
\indent
Speaker, \newline 
\indent
Society of Graduate Students \newline
\indent
sogs.speaker@uwo.ca \newline
\indent 
November 2017


\index{Absenteeism!Committee!Speaker's Ruling|)}